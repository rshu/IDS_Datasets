\section{Dataset}

\subsection{ADFA Linux Dataset (ADFA-LD12)}

\textbf{Cited papers:} ~\cite{creech2014semantic} ~\cite{creech2013generation} \newline
\textbf{Description:} This dataset is generated based on Ubuntu Linux version 11.04, which runs a Linux web server and services with known vulnerabilities. It provides some services such as file sharing, database services, remote access and web server. This dataset recorded  the traces under attacks as described in the following table. \newline

\begin{table}[h]
\begin{tabular}{|l|l|}
\hline
\multicolumn{1}{|c|}{\textbf{Payload/Effect}} & \multicolumn{1}{c|}{\textbf{Attack Vector}} \\ \hline
Password bruteforce & FTP by Hydra \\ \hline
Password bruteforce & SSH by Hydra \\ \hline
Add new superuser & Client side poisoned executable \\ \hline
Java Base Meterpreter & Tiki Wiki vulnerability exploit \\ \hline
Linux Meterpreter payload & Client side poisoned executable \\ \hline
C100 Webshell & \begin{tabular}[c]{@{}l@{}}PHP Remote File Inclusion \\ vulnerability\end{tabular} \\ \hline
\end{tabular}
\caption{Attack Structure}
\end{table}

\textbf{Research Goals:} ADFA-LD12 is designed for anomaly based systems, not for signature recognition IDS. Compared with old datasets such as KDD 98 and KDD 99, this dataset is much more representative of current attacks, and forms a realistic  and relevant metric for IDS performance metrics. Authors in the papers proposed a  new host-based anomaly detection method using discontiguous system call pattern on the ADFA dataset in an attempt to increase detection rate while reducing false alarm rates.\newline
\textbf{Data Structure:} This dataset contains three different data groups, and each group contains raw system call traces. Each training or validation data trace  was collected during normal operation of the host. Traces are generated using the auditd Unix program and then filtered by size. The training data has a size  with range 300 Bytes and 6KB, while the validation data keeps a size between 300 Bytes and 10KB. However, we could not get more details to infer the attributes from the description of the paper.\newline

\begin{table}[h]
\begin{tabular}{|l|c|}
\hline
\multicolumn{1}{|c|}{\textbf{Data Type}} & \textbf{Trace Count} \\ \hline
Normal Training Data & 833 Traces \\ \hline
Normal Validation Data & 4373 Traces \\ \hline
Attack Data & 100 Attacks per Vector \\ \hline
\end{tabular}
\caption{Dataset Structure}
\end{table}

\textbf{Evaluation Measurements:} The validity of the data set was examined by evaluating the performance of several IDS algorithms, i.e., hidden Markov models, the STIDE approach, K-Means clustering, and the K-Nearest Neighbour algorithm, and proposed methods. The result shows that the proposed method achieves a higher detection rate with a lower false alarm rate when compared with other algorithms.\newline

\subsection{KDD Cup 1999 Data}
\textbf{Cited papers:}\newline
\textbf{Description:} The KDD experiments are performed on a Solaris-based system to collect a wide range of data. System calls are generated by processing the BSM audit data. 
\textbf{Research Goals:} \newline
\textbf{Data Structure:} \newline
\textbf{Evaluation Measurements:} \newline

\subsection{Public PCAP file}
\textbf{Cited papers:} \newline
\textbf{Description:} \newline
\textbf{Research Goals:} \newline
\textbf{Data Structure:} \newline
\textbf{Evaluation Measurements:} \newline

\subsection{Cyber Research Center - Datasets CDX Dataset from west point}
\textbf{Cited papers:} \newline
\textbf{Description:} \newline
\textbf{Research Goals:} \newline
\textbf{Data Structure:} \newline
\textbf{Evaluation Measurements:} \newline

\subsection{UNB Canadian Institute for Cybersecurity}

\subsection{HTTP DATASET CSIC 2010}

\subsection{Analyzing Web Traffic ECML/PKDD 2007 Discovery Challenge}

\subsection{CAIDA DDoS Attack}

\subsection{DARPA Intrusion Detection Data Sets}

\subsection{AWID}

\subsection{Coburg Intrusion Detection Data Sets}

\subsection{Comprehensive, Multi-Source Cyber-Security Events}

\subsection{User-Computer Authentication Associations in Time}

\subsection{Unified Host and Network Dataset}

\subsection{Stratosphere IPS}

\subsection{Detecting Malicious URLs}

\subsection{Drebin Dataset}

\subsection{UNM Intrusion Detection Dataset}